%# -*- coding:utf-8 -*-
% Author: Guang Touge(chenqiyuan1012@gmail.com), PhilFan(hw.phil.fan@gmail.com)
% Date: 2025-07-10

\PassOptionsToPackage{quiet}{fontspec}
\documentclass[10pt,aspectratio=169]{beamer}
\usepackage{zju_beamer}
\usepackage{amsmath,amsfonts,amssymb,bm}    % 导入数学公式所需宏包
\usepackage{color}           % 字体颜色支持
\usepackage{graphicx,hyperref,url}
\usepackage{metalogo}
\usepackage{booktabs, float}
\usepackage{tikz}           % 绘图支持
\usetikzlibrary{arrows, positioning}
\graphicspath{{./figures/}} % 指定图片所在文件夹
\zjusectioncolor{blue} % 顶部栏右上角的颜色,blue/black
\catcode`\。=\active
\newcommand{。}{.}
\usepackage[backend=biber, style=ieee, autocite=footnote, citestyle=authortitle]{biblatex}
\addbibresource{references.bib}
% 首页信息设置
\title[卡尔曼滤波算法]{从推算到最优估计:卡尔曼滤波}
\subtitle{——信息工程中的数学建模与应用}

\author[您的姓名]{
  您的姓名 \\ \medskip
  {\small \url{your.email@university.edu.cn}} \\
  {\small 信息与电子工程学院}}

\institute[University]{
  信息工程专业 \\
  数学建模课程小组}

\date[\today]{
  \today}

\begin{document}

%--- 标题页 ---
\begin{frame}
	\titlepage
\end{frame}

%--- 提纲页 ---
\section{提纲}
\begin{frame}
	\frametitle{演讲提纲}
	\tableofcontents
\end{frame}
\AtBeginSection[]
{
	\begin{frame}
		\frametitle{章节导航} % 这一页的标题
		\tableofcontents[currentsection] % 自动生成目录,并高亮当前正在讲的章节
	\end{frame}
}
%================================================================
\section{背景引入}
%================================================================

\begin{frame}
	\frametitle{现实世界的困境:不确定性}
	\begin{columns}
		\column{0.6\textwidth}
		在信息工程与控制领域,我们面临两个永恒的问题:
		\begin{itemize}
			\item \textbf{不准确的模型 (Process Noise)}:
			      \begin{itemize}
				      \item 摩擦力、风阻难以完美建模
				      \item 积分误差随时间累积(Dead Reckoning Drifting)
			      \end{itemize}
			\item \textbf{不准确的观测 (Measurement Noise)}:
			\item GPS 信号多径效应
			\item 传感器热噪声
		\end{itemize}

		\column{0.4\textwidth}
		\begin{block}{核心问题}
			当我们既不能完全相信\textbf{推算},也不能完全相信\textbf{观测}时,如何获知系统的\textbf{真实状态}?
		\end{block}
	\end{columns}
\end{frame}

\begin{frame}
	\frametitle{卡尔曼滤波 (Kalman Filter) 的诞生}
	\begin{itemize}
		\item \textbf{提出者}:Rudolf E. Kalman (1960)
		\item \textbf{核心论文}:\textit{A New Approach to Linear Filtering and Prediction Problems}
		\item \textbf{本质}:
		      \begin{itemize}
			      \item 并不是为了滤除某些频率(不同于低通/高通滤波器)。
			      \item 是一种\textbf{最优递归估计算法} (Optimal Recursive Estimator)。
			      \item 在最小均方误差 (MMSE) 准则下,对系统状态进行最优估计。
		      \end{itemize}
		\item \textbf{应用场景}:阿波罗登月导航、雷达目标追踪、导弹制导、股市预测。
	\end{itemize}
\end{frame}

%================================================================
\section{数学模型构建}
%================================================================

\begin{frame}
	\frametitle{1. 状态方程 (预测模型) 深度解析}
	\begin{equation}
		\bm{x}_k = \underbrace{\bm{F} \bm{x}_{k-1}}_{\text{惯性演变}} + \underbrace{\bm{B} \bm{u}_k}_{\text{控制输入}} + \underbrace{\bm{w}_k}_{\text{过程噪声}}
	\end{equation}

	\begin{itemize}
		\item \textbf{惯性演变 ($\bm{F}\bm{x}_{k-1}$)}:
		      \begin{itemize}
			      \item \textbf{含义}:基于物理定律的自然推演。如果没有外力和干扰,系统下一刻应该在哪?
			      \item \textbf{举例}:根据“路程 = 速度 $\times$ 时间”,由上一刻的位置和速度,推算当前时刻的预计位置。
			      \item $\bm{F}$ (状态转移矩阵) 捕捉了这些确定性的物理规则。
		      \end{itemize}

		\item \textbf{控制输入 ($\bm{B}\bm{u}_k$)}:
		      \begin{itemize}
			      \item \textbf{含义}:由于我们对系统的干预(Control)所引起的状态改变。
			      \item \textbf{举例}:司机踩下油门 ($\bm{u}_k$),通过车辆动力学模型 ($\bm{B}$) 转化为加速度,从而改变车辆的速度和位置。
		      \end{itemize}

		\item \textbf{过程噪声 ($\bm{w}_k$)}:
		      \begin{itemize}
			      \item \textbf{含义}:物理模型中未考虑到的现实扰动(如侧风、路面打滑)。
		      \end{itemize}
	\end{itemize}
\end{frame}

\begin{frame}
	\frametitle{2. 观测方程 (测量模型) 深度解析}
	\begin{equation}
		\bm{z}_k = \underbrace{\bm{H} \bm{x}_k}_{\text{理想观测}} + \underbrace{\bm{v}_k}_{\text{测量噪声}}
	\end{equation}

	\begin{itemize}
		\item \textbf{理想观测 ($\bm{H}\bm{x}_k$)}:
		      \begin{itemize}
			      \item \textbf{含义}:状态空间的映射与降维。如果传感器完美无瑕,它应该读到什么数值?
			      \item \textbf{矩阵作用}:系统状态 $\bm{x}$ 可能包含 [位置, 速度, 加速度] ($3 \times 1$),但 GPS 传感器 $\bm{z}$ 只能读取 [位置] ($1 \times 1$)。
			      \item $\bm{H}$ (观测矩阵) 就是一个提取器:例如 $\begin{bmatrix} 1 & 0 & 0 \end{bmatrix} \times \begin{bmatrix} pos \\ vel \\ acc \end{bmatrix} = pos$。
		      \end{itemize}

		\item \textbf{测量噪声 ($\bm{v}_k$)}:
		      \begin{itemize}
			      \item \textbf{含义}:传感器自身的“不靠谱”程度。包括电子热噪声、量化误差或环境干扰。
			      \item 我们假设它是白噪声,服从 $N(0, \bm{R})$。
		      \end{itemize}
	\end{itemize}
\end{frame}

\begin{frame}
	\frametitle{符号含义解析}
	\small
	\begin{table}
		\centering
		\begin{tabular}{cll}
			\toprule
			符号         & 含义                   & 维度           \\
			\midrule
			$\bm{x}_k$ & $k$时刻的状态向量 (位置, 速度等) & $n \times 1$ \\
			$\bm{F}$   & 状态转移矩阵 (物理规律)        & $n \times n$ \\
			$\bm{u}_k$ & 外部控制量 (如加速度输入)       & $l \times 1$ \\
			$\bm{z}_k$ & $k$时刻的观测向量 (传感器读数)   & $m \times 1$ \\
			$\bm{H}$   & 观测矩阵 (映射关系)          & $m \times n$ \\
			\bottomrule
		\end{tabular}
	\end{table}

	\vspace{0.5cm}
	\textbf{关键假设:} 噪声服从高斯分布 (Gaussian Distribution)
	\begin{itemize}
		\item 过程噪声 $\bm{w}_k \sim N(0, \bm{Q})$
		\item 测量噪声 $\bm{v}_k \sim N(0, \bm{R})$
	\end{itemize}
\end{frame}

%================================================================
\section{核心算法}
%================================================================

\begin{frame}
	\frametitle{算法流程总览}
	卡尔曼滤波是一个\textbf{“预测 (Predict) —— 更新 (Update)”} 的循环过程。
	\vspace{0.5cm}

	\begin{center}
		\fbox{\parbox{0.8\textwidth}{\centering
				\textbf{Step 1: 预测} \\
				(根据上一刻状态,猜这一刻在哪) \\
				$\downarrow$ \\
				\textbf{Step 2: 计算卡尔曼增益} \\
				(决定信模型多一点,还是信传感器多一点) \\
				$\downarrow$ \\
				\textbf{Step 3: 更新} \\
				(结合观测值,修正刚才的猜测)
			}}
	\end{center}
\end{frame}

\begin{frame}
	\frametitle{第一阶段:时间更新 (预测)}
	在还未获得传感器数据之前,基于物理模型进行的\textbf{先验估计}:

	\begin{alertblock}{1. 状态预测}
		\begin{equation}
			\hat{\bm{x}}_k^- = \bm{F} \hat{\bm{x}}_{k-1} + \bm{B} \bm{u}_k
		\end{equation}
		\small 注:上标 $^-$ 表示先验 (A Priori) 估计。
	\end{alertblock}

	\begin{alertblock}{2. 协方差预测 (误差传播)}
		\begin{equation}
			\bm{P}_k^- = \bm{F} \bm{P}_{k-1} \bm{F}^T + \bm{Q}
		\end{equation}
		\small $\bm{P}$ 代表估计的不确定性。加上 $\bm{Q}$ 表示预测过程引入了新的不确定性。
	\end{alertblock}
\end{frame}

\begin{frame}
	\frametitle{关键步骤:卡尔曼增益 (Kalman Gain) —— 1. 思路篇}

	\begin{block}{核心思想:不确定性的权衡}
		卡尔曼增益 $\bm{K}_k$ 的本质是一个\textbf{动态权衡因子 (Weighting Factor)}。
		它时刻在问一个问题:
		\vspace{0.3cm}
		\begin{center}
			\textit{“现在的预测误差 ($\bm{P}_k^-$) 和 观测误差 ($\bm{R}$),哪一个更小?”}
		\end{center}
	\end{block}

	\begin{itemize}
		\item \textbf{逻辑:} 谁的方差小(信息量大、更可信),我就偏向谁。
		\item \textbf{动态性:}
		      \begin{itemize}
			      \item 这不是一个固定的常数(如互补滤波中的 $\alpha=0.98$)。
			      \item 它随时间变化,每一次迭代都会根据上一次的置信度自动调整。
		      \end{itemize}
	\end{itemize}
\end{frame}

\begin{frame}
	\frametitle{关键步骤:卡尔曼增益 (Kalman Gain) —— 2. 矩阵直觉篇}
	为什么公式里有那么多 $\bm{H}^T$ 和逆矩阵?
	\begin{equation*}
		\bm{K}_k = \underbrace{\bm{P}_k^- \bm{H}^T}_{\text{相关性}} \underbrace{(\bm{H} \bm{P}_k^- \bm{H}^T + \bm{R})^{-1}}_{\text{归一化}}
	\end{equation*}
	\begin{itemize}
		\item \textbf{空间投影问题}:
		      \begin{itemize}
			      \item 状态误差 $\bm{P}_k^-$ 在\textbf{状态空间} (n维)。
			      \item 观测误差 $\bm{R}$ 在\textbf{测量空间} (m维)。
			      \item 我们不能直接把它们相加,必须把 $\bm{P}_k^-$ 投影到测量空间。
		      \end{itemize}
		\item \textbf{$\bm{H} \bm{P}_k^- \bm{H}^T$}:这是预测误差在\textbf{测量空间}的投影。即“如果在状态空间有这么多不确定性,对应到传感器读数上会有多大波动”。
		\item \textbf{分母求逆}:$(\bm{S})^{-1}$ 相当于除以“总不确定性”。
	\end{itemize}
\end{frame}

\begin{frame}
	\frametitle{关键步骤:卡尔曼增益 (Kalman Gain) —— 3. 算法篇}

	\begin{block}{3. 计算卡尔曼增益}
		\begin{equation}
			\bm{K}_k = \bm{P}_k^- \bm{H}^T (\bm{H} \bm{P}_k^- \bm{H}^T + \bm{R})^{-1}
		\end{equation}
	\end{block}

	\begin{columns}
		\column{0.5\textwidth}
		\textbf{复杂矩阵运算的直观类比:}\\
		假设是一维标量系统($H=1$):
		$$ K_k = \frac{P_k^-}{P_k^- + R} $$
		\column{0.5\textwidth}
		\textbf{含义解析:}
		\begin{itemize}
			\item \textbf{分子} ($P_k^-$):预测误差的方差。
			\item \textbf{分母} ($P_k^- + R$):总误差(预测+观测)。
			\item \textbf{结果}:预测误差在总误差中的占比。
		\end{itemize}
	\end{columns}
\end{frame}

\begin{frame}
	\frametitle{关键步骤:卡尔曼增益 (Kalman Gain) —— 4. 效果篇}
	通过极限情况,我们可以清晰看到 $\bm{K}_k$ 的调节机制:

	\vspace{0.5cm}
	\begin{columns}
		\column{0.48\textwidth}
		\begin{alertblock}{情况 A: 传感器极准 ($\bm{R} \to 0$)}
			\begin{itemize}
				\item 此时 $K_k \approx \frac{P}{P} = 1$ (最大)。
				\item 状态更新变为:\\
				      $\hat{x}_k = \hat{x}_k^- + 1 \cdot (z_k - \hat{x}_k^-) = z_k$
				\item \textbf{结论}:完全信任观测值,忽略模型预测。
			\end{itemize}
		\end{alertblock}

		\column{0.48\textwidth}
		\begin{exampleblock}{情况 B: 模型极准 ($\bm{P}_k^- \to 0$)}
			\begin{itemize}
				\item 此时 $K_k \approx \frac{0}{0+R} = 0$ (最小)。
				\item 状态更新变为:\\
				      $\hat{x}_k = \hat{x}_k^- + 0 = \hat{x}_k^-$
				\item \textbf{结论}:完全信任模型预测,忽略传感器波动。
			\end{itemize}
		\end{exampleblock}
	\end{columns}
\end{frame}

\begin{frame}
	\frametitle{第二阶段:完成测量更新}
	算出增益 $\bm{K}_k$ 后,我们就可以融合数据并更新系统信心。

	\begin{block}{4. 状态更新 (后验估计)}
		\begin{equation}
			\hat{\bm{x}}_k = \hat{\bm{x}}_k^- + \bm{K}_k \underbrace{(\bm{z}_k - \bm{H} \hat{\bm{x}}_k^-)}_{\text{残差 (Innovation)}}
		\end{equation}
		\small \textbf{最优估计 = 预测 + 修正量}。残差代表了“观测值与预测值的偏差”。
	\end{block}

	\begin{block}{5. 协方差更新}
		\begin{equation}
			\bm{P}_k = (\bm{I} - \bm{K}_k \bm{H}) \bm{P}_k^-
		\end{equation}
		\small 更新系统的不确定性。因为融合了新信息,通常 $\bm{P}_k < \bm{P}_k^-$,表示我们对系统的状态越来越有把握。
	\end{block}
\end{frame}

%================================================================
\section{数值推演实例}
%================================================================

\begin{frame}
	\frametitle{场景设定:无人机悬停高度估计}
	\begin{columns}
		\column{0.5\textwidth}
		\textbf{场景:}
		\begin{itemize}
			\item 我们想知道无人机的真实高度 $x$。
			\item 无人机正在悬停,理论上不动,但有风吹(过程噪声)。
			\item 使用气压计测量高度,但有误差(测量噪声)。
		\end{itemize}

		\column{0.5\textwidth}
		\textbf{参数初始化 ($t=0$):}
		\begin{itemize}
			\item \textbf{初始估计} $\hat{x}_0 = 10m$。
			\item \textbf{初始不确定性} $P_0 = 5$ (不太确信)。
		\end{itemize}
		\textbf{系统参数:}
		\begin{itemize}
			\item 状态转移 $F=1$ (不动)。
			\item 过程噪声 $Q=1$。
			\item 测量噪声 $R=4$ (传感器一般)。
			\item 观测矩阵 $H=1$ (直接测高度)。
		\end{itemize}
	\end{columns}
\end{frame}

\begin{frame}
	\frametitle{Step 1: 预测 (Prediction) @ t=1}
	还没看传感器,先根据物理规律猜。

	\begin{block}{1. 状态预测}
		$$ \hat{x}_1^- = F \hat{x}_0 = 1 \times 10 = 10 $$
		\small (我猜它还在 10m 处)
	\end{block}

	\begin{block}{2. 协方差预测}
		$$ P_1^- = F P_0 F^T + Q = 1 \times 5 \times 1 + 1 = 6 $$
		\small (如果不看传感器,不确定性从 5 增加到了 6,因为风在吹)
	\end{block}
\end{frame}

\begin{frame}
	\frametitle{Step 2: 计算卡尔曼增益 @ t=1}
	现在传感器读数来了:$\bm{z}_1 = 12m$。

	\begin{block}{3. 计算增益 K}
		$$ K_1 = \frac{P_1^- H^T}{H P_1^- H^T + R} = \frac{6 \times 1}{1 \times 6 \times 1 + 4} = \frac{6}{10} = 0.6 $$
	\end{block}

	\textbf{解读:}
	\begin{itemize}
		\item $K=0.6$ 意味着我们把 \textbf{60\%} 的信任给了观测值(残差),\textbf{40\%} 保留给预测值。
		\item 为什么信观测多一点?因为预测误差方差(6) 大于 测量噪声方差(4)。
	\end{itemize}
\end{frame}

\begin{frame}
	\frametitle{Step 3: 更新 (Update) @ t=1}
	融合预测和观测,得到最优结果。

	\begin{alertblock}{4. 状态更新}
		$$ \hat{x}_1 = \hat{x}_1^- + K_1 (z_1 - \hat{x}_1^-) $$
		$$ \hat{x}_1 = 10 + 0.6 \times (12 - 10) = 10 + 1.2 = \mathbf{11.2m} $$
	\end{alertblock}

	\begin{block}{5. 协方差更新}
		$$ P_1 = (1 - K_1 H) P_1^- = (1 - 0.6) \times 6 = 0.4 \times 6 = \mathbf{2.4} $$
	\end{block}

	\textbf{结果:}
	\begin{itemize}
		\item 估计值 11.2m 介于 10m 和 12m 之间,偏向观测值。
		\item 不确定性从 6 骤降至 2.4。滤波让我们更自信了!
	\end{itemize}
\end{frame}

%================================================================
\section{工程应用与仿真}
%================================================================

\begin{frame}
	\frametitle{仿真案例:一维小车位置追踪}
	\textbf{场景假设:}
	\begin{itemize}
		\item 小车做匀速运动,速度 $v=1m/s$。
		\item 过程噪声 $\bm{Q}$:路面颠簸导致速度微小波动。
		\item 测量噪声 $\bm{R}$:GPS 定位误差 $\pm 5m$。
	\end{itemize}

\end{frame}
\begin{frame}
	\frametitle{仿真案例:一维小车位置追踪}
	% \vspace{-5cm}
	\begin{figure}[H]
		\centering
		\includegraphics[width = .5\textwidth]{Kal.png}
		\caption{利用Matlab模拟卡尔曼滤波}
	\end{figure}
\end{frame}
\begin{frame}
	\frametitle{参数 $\bm{Q}$ 与 $\bm{R}$ 的调节策略}
	在信息工程实践中,$\bm{F}$ 和 $\bm{H}$ 通常由物理系统决定,但 $\bm{Q}$ 和 $\bm{R}$ 是调试的关键。

	\begin{columns}
		\column{0.5\textwidth}
		\begin{block}{信任模型 ($\bm{Q}$ 小, $\bm{R}$ 大)}
			\begin{itemize}
				\item 滤波结果非常平滑
				\item 对突变反应滞后 (Lag)
			\end{itemize}
		\end{block}

		\column{0.5\textwidth}
		\begin{block}{信任观测 ($\bm{Q}$ 大, $\bm{R}$ 小)}
			\begin{itemize}
				\item 紧跟观测值变化
				\item 引入较多噪声 (Noise)
			\end{itemize}
		\end{block}
	\end{columns}

	\vspace{0.5cm}
	\textbf{结论:} 调参本质上是在\textbf{响应速度}与\textbf{平滑度}之间做权衡。
\end{frame}

\begin{frame}
	\frametitle{进阶:多传感器融合 (Sensor Fusion)}
	卡尔曼滤波是多传感器融合的基石。

	\textbf{案例:无人机姿态解算}
	\begin{itemize}
		\item \textbf{陀螺仪 (Gyro)}:短时精度高,长时有积分漂移。
		\item \textbf{加速度计 (Accel)}:静态精度高,动态噪声大。
		\item \textbf{融合方案}:利用 Kalman Filter 融合两者数据。
		      \begin{equation*}
			      \text{Optimal Angle} = K \cdot \text{Accel} + (1-K) \cdot \text{Gyro}
		      \end{equation*}
	\end{itemize}
\end{frame}

%================================================================
\section{扩展卡尔曼滤波 (EKF)}
%================================================================

\begin{frame}
	\frametitle{1. 线性系统的局限性}
	\begin{alertblock}{标准 KF 的致命弱点}
		标准卡尔曼滤波假设系统是\textbf{线性 (Linear)} 的:
		$$ \bm{x}_k = \bm{F}\bm{x}_{k-1} \quad \text{和} \quad \bm{z}_k = \bm{H}\bm{x}_k $$
	\end{alertblock}

	\textbf{然而,现实世界充满了非线性:}
	\begin{itemize}
		\item \textbf{机器人运动}:$x_{new} = x_{old} + v \cdot \cos(\theta) \cdot \Delta t$ (包含三角函数)
		\item \textbf{雷达观测}:$r = \sqrt{x^2 + y^2}$ (包含平方根)
	\end{itemize}

	\vspace{0.2cm}
	\textbf{问题:} 高斯分布经过非线性变换后,\textbf{不再是高斯分布}(形状会扭曲),导致标准 KF 的公式失效。
\end{frame}

\begin{frame}
	\frametitle{2. 解决思路:线性化 (Linearization)}
	既然非线性太难处理,我们能不能在\textbf{局部}把它看作线性的?

	\begin{block}{泰勒级数展开 (Taylor Series Expansion)}
		对于非线性函数 $f(x)$,我们在估计点 $\hat{x}$ 附近做一阶展开:
		\begin{equation}
			f(x) \approx f(\hat{x}) + \underbrace{\frac{\partial f}{\partial x}\bigg|_{x=\hat{x}}}_{\text{切线斜率}} \cdot (x - \hat{x})
		\end{equation}
	\end{block}

	\begin{itemize}
		\item 我们抛弃高阶项,只保留一阶导数。
		\item 用\textbf{切线}来近似\textbf{曲线}。
	\end{itemize}
\end{frame}

\begin{frame}
	\frametitle{3. 核心工具:雅可比矩阵 (Jacobian Matrix)}
	在多维系统中,导数变成了\textbf{雅可比矩阵}。它是 EKF 的核心。

	假设状态转移函数为 $\bm{x}_k = f(\bm{x}_{k-1}, \bm{u}_k)$,则雅可比矩阵 $\bm{F}_k$ 为:

	\begin{equation}
		\bm{F}_k = \frac{\partial f}{\partial \bm{x}} =
		\begin{bmatrix}
			\frac{\partial f_1}{\partial x_1} & \frac{\partial f_1}{\partial x_2} & \cdots \\
			\frac{\partial f_2}{\partial x_1} & \frac{\partial f_2}{\partial x_2} & \cdots \\
			\vdots                            & \vdots                            & \ddots
		\end{bmatrix}
	\end{equation}

	\textbf{物理意义:} 描述了输入微小变化如何影响输出的每一个维度。
\end{frame}

\begin{frame}
	\frametitle{4. EKF 算法流程的变化 (对比 KF)}
	EKF 的步骤与 KF 几乎一样,区别在于\textbf{如何传递均值}和\textbf{如何传递协方差}。

	\begin{table}
		\centering
		\small
		\begin{tabular}{l|c|c}
			\toprule
			步骤             & 标准 KF (线性)                                            & 扩展 KF (非线性)                                                                         \\
			\midrule
			\textbf{状态预测}  & $\bm{x}^- = \bm{F}\bm{x}$                             & $\bm{x}^- = \mathbf{f}(\bm{x}, \bm{u})$ \textcolor{blue}{(直接代入函数)}                  \\
			\textbf{协方差预测} & $\bm{P}^- = \bm{F}\bm{P}\bm{F}^T + \bm{Q}$            & $\bm{P}^- = \mathbf{F_k}\bm{P}\mathbf{F_k}^T + \bm{Q}$ \textcolor{red}{(用雅可比)}      \\
			\midrule
			\textbf{卡尔曼增益} & $\bm{K} = \bm{P}^-\bm{H}^T(\dots)^{-1}$               & $\bm{K} = \bm{P}^-\mathbf{H_k}^T(\mathbf{H_k}\bm{P}^-\mathbf{H_k}^T + \bm{R})^{-1}$ \\
			\textbf{状态更新}  & $\bm{x} = \bm{x}^- + \bm{K}(\bm{z} - \bm{H}\bm{x}^-)$ & $\bm{x} = \bm{x}^- + \bm{K}(\bm{z} - \mathbf{h}(\bm{x}^-))$                         \\
			\bottomrule
		\end{tabular}
	\end{table}
\end{frame}

\begin{frame}
	\frametitle{5. 经典案例:雷达追踪 (Radar Tracking)}
	\begin{columns}
		\column{0.5\textwidth}
		\textbf{状态向量 (直角坐标系):} \\
		$\bm{x} = [p_x, p_y, v_x, v_y]^T$ \\
		(飞机的位置和速度)

		\vspace{0.3cm}
		\textbf{观测向量 (极坐标系):} \\
		$\bm{z} = [\rho, \phi, \dot{\rho}]^T$ \\
		(雷达测量的距离、角度、径向速度)

		\column{0.5\textwidth}
		\textbf{非线性观测函数 $h(\bm{x})$:}
		\begin{itemize}
			\item 距离:$\rho = \sqrt{p_x^2 + p_y^2}$
			\item 角度:$\phi = \arctan(p_y / p_x)$
		\end{itemize}
	\end{columns}

	\vspace{0.5cm}
	\begin{block}{问题}
		$p_x$ 和 $p_y$ 是状态,但在观测方程里被平方和求根了。这就是典型的非线性!必须求雅可比矩阵 $\bm{H}_j$。
	\end{block}
\end{frame}

\begin{frame}
	\frametitle{6. 雷达雅可比矩阵的计算}
	我们需要对 $h(\bm{x})$ 求偏导数来得到 $\bm{H}_j$:

	\small
	\begin{equation}
		\bm{H}_j = \frac{\partial h}{\partial \bm{x}} =
		\begin{bmatrix}
			\frac{\partial \rho}{\partial p_x} & \frac{\partial \rho}{\partial p_y} & 0 & 0 \\
			\frac{\partial \phi}{\partial p_x} & \frac{\partial \phi}{\partial p_y} & 0 & 0
		\end{bmatrix}
		=
		\begin{bmatrix}
			\frac{p_x}{\sqrt{p_x^2+p_y^2}} & \frac{p_y}{\sqrt{p_x^2+p_y^2}} & 0 & 0 \\
			\frac{-p_y}{p_x^2+p_y^2}       & \frac{p_x}{p_x^2+p_y^2}        & 0 & 0
		\end{bmatrix}
	\end{equation}

	\vspace{0.3cm}
	\textbf{意义:} 这个矩阵把\textbf{位置的不确定性}(直角坐标误差),投影到了\textbf{观测的不确定性}(距离和角度误差)上。
\end{frame}

\begin{frame}
	\frametitle{7. EKF 的优缺点分析}

	\begin{block}{优点 (Why use it?)}
		\begin{itemize}
			\item \textbf{事实标准}:目前导航系统(GPS/IMU 融合)、机器人定位 (SLAM) 的首选基础算法。
			\item \textbf{计算量适中}:比粒子滤波 (Particle Filter) 快得多。
		\end{itemize}
	\end{block}

	\begin{alertblock}{缺点 (Watch out!)}
		\begin{itemize}
			\item \textbf{发散风险}:如果线性化点选得不好(初始估计太差),泰勒展开误差会很大,导致滤波器发散。
			\item \textbf{繁琐的雅可比}:对于复杂的系统,手算雅可比矩阵非常痛苦且容易出错。
		\end{itemize}
	\end{alertblock}
\end{frame}

\begin{frame}
	\frametitle{8. 超越 EKF:无迹卡尔曼滤波 (UKF)}
	当非线性极其严重时,EKF 的线性化误差无法接受。

	\begin{columns}
		\column{0.6\textwidth}
		\textbf{UKF (Unscented Kalman Filter) 思路:}
		\begin{itemize}
			\item 不去线性化函数(不算雅可比)。
			\item 而是\textbf{近似概率分布}。
			\item 选取几个关键点(Sigma Points),把它们扔进非线性函数里算一算,然后再算出新的均值和方差。
		\end{itemize}

		\column{0.4\textwidth}
		\begin{block}{核心理念}
			\small
			“近似概率分布比近似非线性函数要容易得多。”
			\\ —— Julier \& Uhlmann
		\end{block}
	\end{columns}
\end{frame}

%================================================================
%================================================================
\section{前沿探索}
%================================================================

% --- 第一部分:基于顶级综述的算法演进 (Khodarahmi et al., 2023) ---

\begin{frame}
	\frametitle{1. 算法演进全景 (Based on Top Review)}
	\small
	卡尔曼滤波已发展为庞大的家族。针对不同场景,主流算法的特性对比如下\footcite{khodarahmi_review_2023}:

	% 表格:主流算法对比
	\begin{table}
		\centering
		\renewcommand{\arraystretch}{1.2}
		\resizebox{0.95\textwidth}{!}{
			\begin{tabular}{l l l l}
				\toprule
				\textbf{算法}       & \textbf{核心机制}      & \textbf{优势 (Pros)}   & \textbf{适用场景} \\
				\midrule
				\textbf{Basic KF} & 线性递归最小二乘           & 计算极快,线性最优            & 卫星导航,稳态控制     \\
				\textbf{EKF}      & 泰勒展开线性化 ($\bm{J}$) & 工业标准,适用性广            & 机器人定位 (SLAM)  \\
				\textbf{UKF}      & 无迹变换 (Sigma点)      & 无需导数,精度更高            & 复杂非线性系统       \\
				\textbf{IMM}      & 多模型概率交互            & \textbf{搞定机动目标},平滑切换 & 导弹/无人机追踪      \\
				\bottomrule
			\end{tabular}
		}
	\end{table}
\end{frame}

% --- 第二部分:KF 与神经网络的结合 (Feng et al., 2023) ---

\begin{frame}
	\frametitle{2. 突破瓶颈:混合模型 (Hybrid Models) 的诞生}

	\begin{columns}
		\column{0.6\textwidth}
		% 左侧:放流程图(占位符)
		\centering
		\begin{figure}
			% 这里的 hybrid.png 就是您那张流程图
			% keepaspectratio 保持长宽比,width 设置为栏宽的 95%
			\includegraphics[width=0.95\textwidth, keepaspectratio]{hybrid.png}
			\caption{状态估计技术融合路线图\footcite{feng_review_2023}}
		\end{figure}

		\column{0.4\textwidth}
		% 右侧:文字解说
		\textbf{为何要融合?}
		\begin{itemize}
			\item \textbf{传统 KF}:依赖精确物理模型,难以处理未知环境。
			\item \textbf{神经网络 (NN)}:强大的数据拟合能力,但缺乏物理约束。
		\end{itemize}

		\vspace{0.3cm}
		\textbf{融合的三种模式:}
		\begin{enumerate}
			\small
			\item \textbf{串联}:互为预处理/后处理。
			\item \textbf{训练}:KF 优化 NN 权重。
			\item \textbf{辅助 (Aided)}:\textcolor{red}{\textbf{NN 实时估计 KF 参数 ($\bm{Q}, \bm{R}$)}}。
		\end{enumerate}
	\end{columns}
\end{frame}
\begin{frame}
	\frametitle{深度思考:卡尔曼滤波 vs 机器学习}
	\begin{columns}
		% 左栏:卡尔曼滤波 (Model-Based)
		\column{0.48\textwidth}
		\begin{block}{卡尔曼滤波 (KF)}
			\centering \textbf{“理性的物理学家”}
			\begin{itemize}
				\item \textbf{驱动核心}:物理模型 ($\bm{F}, \bm{H}$) + 概率统计。
				\item \textbf{透明度}:\textbf{白盒}。每一步都有明确物理含义,可解释性强。
				\item \textbf{优势}:小样本即可工作,不仅给结果,还给\textbf{置信度} ($\bm{P}$)。
				\item \textbf{劣势}:模型必须已知且准确。
			\end{itemize}
		\end{block}

		% 右栏:机器学习 (Data-Driven)
		\column{0.48\textwidth}
		\begin{exampleblock}{机器学习 (ML/NN)}
			\centering \textbf{“经验丰富的工匠”}
			\begin{itemize}
				\item \textbf{驱动核心}:海量数据 + 拟合映射。
				\item \textbf{透明度}:\textbf{黑盒}。内部权重难以解释。
				\item \textbf{优势}:能拟合极其复杂的非线性关系,无需懂物理机理。
				\item \textbf{劣势}:数据饥渴,对未见过的场景泛化能力弱。
			\end{itemize}
		\end{exampleblock}
	\end{columns}

	\vspace{0.4cm}
	\centering
	\textbf{融合的哲学:}
	\fcolorbox{blue}{white}{用 KF 的逻辑框架约束 ML 的发散,用 ML 的拟合能力弥补 KF 的模型缺陷。}
\end{frame}
% --- 第三部分:锂电池 SOC 估计应用 ---

\begin{frame}
	\frametitle{3. 落地应用:锂电池 SOC 状态估计}
	\small
	\textbf{背景:} SOC (State of Charge) 是电池管理系统的核心。估算错误会导致过充/过放,造成永久性损伤。

	\vspace{0.3cm}
	\textbf{核心技术栈 (Tech Stack):}
	\begin{columns}[t]
		\column{0.48\textwidth}
		\begin{block}{1. 非线性滤波 (NLKFs)}
			用于处理电池高度非线性的电压特性:
			\begin{itemize}
				\item \textbf{EKF / AEKF} (自适应扩展卡尔曼)
				\item \textbf{UKF / AUKF} (自适应无迹卡尔曼)
			\end{itemize}
		\end{block}

		\column{0.48\textwidth}
		\begin{block}{2. 在线参数辨识}
			实时更新电池模型参数(如内阻):
			\begin{itemize}
				\item \textbf{RLS} (递归最小二乘法)
				\item \textbf{PRBM} (多项式回归模型)
			\end{itemize}
		\end{block}
	\end{columns}

	\vspace{0.4cm}
	\textbf{实验结论:}
	在宽温域 (-5\textcelsius $\sim$ 45\textcelsius) 测试中,以下组合表现最佳:
	\begin{center}
		\fcolorbox{red}{white}{\textbf{PRBM-AUKF} \quad \& \quad \textbf{RLS-AUKF}}
	\end{center}
	\footnotesize \textit{* AUKF (Adaptive UKF) 展现了比传统 EKF 更高的精度和鲁棒性。\footcite{hossain_kalman_2022}}
\end{frame}


%================================================================
\section{结论}
%================================================================

\begin{frame}
	\frametitle{总结}

	\begin{enumerate}
		\item \textbf{有效性}:卡尔曼滤波利用了系统的\textbf{冗余信息}(模型预测 + 传感器观测),从而降低了整体的不确定性。
		\item \textbf{高效性}:递归算法,只需存储上一刻状态,非常适合\textbf{嵌入式系统}及\textbf{实时控制}。
		\item \textbf{局限性}:
		      \begin{itemize}
			      \item 假设系统是线性的 (Linear)。
			      \item 假设噪声是高斯的 (Gaussian)。
		      \end{itemize}
		\item \textbf{拓展}:面对非线性系统,可以使用\textbf{扩展卡尔曼滤波 (EKF)}。
	\end{enumerate}
\end{frame}

\section{参考文献}
\begin{frame}{参考文献}
	\printbibliography
\end{frame}

\begin{frame}
	\centering
	\Huge
	\textcolor{structure}{谢谢大家!}\\
	\large Q \& A
\end{frame}

\end{document}